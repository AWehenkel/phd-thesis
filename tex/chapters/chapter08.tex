\chapter{Conclusion}\label{ch:08}
At first glance, the way nature drives the world around us appears chaotic. Still, some perspectives reveal patterns in this illusive disorder. The ability to discover and exploit these structures is Intelligence. Encoding these structures into mathematical objects called models eventually reduces intelligent reasoning to a list of computing operations and gives rise to artificial intelligence. This dissertation studies the automatic discovery of models, intelligence that produces intelligence. In particular, we argue for a probabilistic approach to modelling that naturally benefits from modern computers, data, and existing knowledge.

%  is the field of science that study the intelligence of human-made objects' behaviour. Intelligent systems are the ones that produce responses that inherits from the structure we perceive as humans. This thesis stands for a probabilistic approach to modelling. Our aim was to prove the effectiveness
%
% This thesis has emphasised the relevance of a probabilistic prospect of modelling.
% Our world is driven by effects whose understanding is out of reach. Yet, this apparent chaos eventually disapear if we take the right viewpoint. things that exhibit some kind of structure. This thesis was about
% Making bets about the future is always a bad
%
%
% 0) One sentence that says that this is a tiny piece in an ocean of research in Machine learning.
% One sentence that says what I did here.
%
% 1) About the contribution of the background
%
% 2) About the contribution of uninformed probabilistic models'
% a) Look back at what we did
%
% b) Look back at what are the exciting development there.
%
% 3) About the contribution of informed models
% a) Look back at what we did
%
% b) Look back at what are the exciting development there.
%
% 4) A final note on the challenges that remain in probabilistic modelling.
% The role it shall have in the future and why it is powerful think framework.
% This is a two fold tools. It allows to think and it allows to do concrete things.
% It is rare for a framework to be as powerful as that. It is why we must continue to develop it and foster its impact on our lives.

%
% \section{Summary}
% - Uninformed vs informed models
% - Combining models is nice.
% - All the advice we should remember flows.
%   1) Do not use to many steps without a good reasons, it does not help expressivity.
%   2) If possible use expressive 1D transformation such as monotonic ones.
%   3) But also bias the learning toward reasonable models - embed as much domain knowledge as you can.
%
%
% \section{The future of deep probabilistic modelling}
% \subsection{Computing complexity}
% - training
% - evaluation
%
% \subsection{Potential applications}
%
% \subsection{Informed models}
% - Automatic model discovery within simulators
% - Use of language models to express hypothesis more easily
% - SBI and data oriented models
%
% ---- Random thoughts
% - Bayesian treatments of deep probabilistic models
% -
