\chapter{Deep Latent Variables Generative Models}\label{ch:03}

\begin{remark}{Outline}
Among likelihood-based approaches for deep generative modelling, variational autoencoders (VAEs) offer scalable amortized posterior inference and fast sampling. However, VAEs are also more and more outperformed by competing models such as normalizing flows (NFs), deep-energy models, or the new denoising diffusion probabilistic models (DDPMs).
In this preliminary work, we improve VAEs by demonstrating how DDPMs can be used for modelling the prior distribution of the latent variables. The diffusion prior model improves upon Gaussian priors of classical VAEs and is competitive with NF-based priors.
Finally, we hypothesize that hierarchical VAEs could similarly benefit from the enhanced capacity of diffusion priors.
\end{remark}
\section{Prologue}
\\
- A word on the role of this part of the manuscript. Focus on the expressivity of deep probablistic models.

\subsection{Retrospective state of the art}

\paragraph{Variational auto-encoder}

\paragraph{Diffusion models}

\section{The paper: Diffusion Priors In Variational Autoencoders}

\subsection{Author contributions}

\subsection{Reading tips}

\subsection{Minor corrections}

\includepdf[pages=-]{papers/innf_latent_diffusion.pdf}

\section{Epilogue}
\subsection{The practitioner's eye}

\subsection{What happened since then?}
- Work on continuous diffusion models and combining Enervy based models and VAES.
https://arxiv.org/pdf/2206.05895.pdf
- Diffusion models largely used for high fidelity image synthesis.
