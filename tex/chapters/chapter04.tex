\chapter{Limitations of Affine Normalizing Flows}\label{ch:04}

\begin{remark}{Outline}
Normalizing flows have emerged as an important family of deep neural networks for modelling complex probability distributions.
In this note, we revisit their coupling and autoregressive transformation layers as probabilistic graphical models and show that they reduce to Bayesian networks  with a pre-defined topology and a learnable density at each node.
From this new perspective, we provide three results.
First, we show that stacking multiple transformations in a normalizing flow relaxes independence assumptions and entangles the model distribution.
Second, we show that a fundamental leap of capacity emerges when the depth of affine flows exceeds 3 transformation layers.
Third, we prove the non-universality of the affine normalizing flow, regardless of its depth.
\end{remark}

\section{Prologue}
- This chapter makes connections between NFs and BNs which allow to better understand  the former. We derive properties that have practical consequences on the design of models.
- Study the expressivity of NFs. Show their limitation. 
\section{The paper: You say Normalizing Flows I see Bayesian Networks}

\subsection{Author contributions}

\subsection{Reading tips}

\subsection{Minor corrections}
\includepdf[pages=-]{papers/innf.pdf}

\section{Epilogue}

\subsection{The practitioner's eye}

\subsection{What happened since then?}
- This work has not gained in visibility enough and Unfortunately it still seems many practicitoners just stack flow steps.
- It would be interesting to understand why it might still be intetesting to stack steps.
- This is strongly related to the paper http://proceedings.mlr.press/v130/behrmann21a/behrmann21a.pdf
